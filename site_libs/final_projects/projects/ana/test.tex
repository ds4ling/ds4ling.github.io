\documentclass[man]{apa6}

\usepackage{amssymb,amsmath}
\usepackage{ifxetex,ifluatex}
\usepackage{fixltx2e} % provides \textsubscript
\ifnum 0\ifxetex 1\fi\ifluatex 1\fi=0 % if pdftex
  \usepackage[T1]{fontenc}
  \usepackage[utf8]{inputenc}
\else % if luatex or xelatex
  \ifxetex
    \usepackage{mathspec}
    \usepackage{xltxtra,xunicode}
  \else
    \usepackage{fontspec}
  \fi
  \defaultfontfeatures{Mapping=tex-text,Scale=MatchLowercase}
  \newcommand{\euro}{€}
\fi
% use upquote if available, for straight quotes in verbatim environments
\IfFileExists{upquote.sty}{\usepackage{upquote}}{}
% use microtype if available
\IfFileExists{microtype.sty}{\usepackage{microtype}}{}

% Table formatting
\usepackage{longtable, booktabs}
\usepackage{lscape}
% \usepackage[counterclockwise]{rotating}   % Landscape page setup for large tables
\usepackage{multirow}		% Table styling
\usepackage{tabularx}		% Control Column width
\usepackage[flushleft]{threeparttable}	% Allows for three part tables with a specified notes section
\usepackage{threeparttablex}            % Lets threeparttable work with longtable

% Create new environments so endfloat can handle them
% \newenvironment{ltable}
%   {\begin{landscape}\begin{center}\begin{threeparttable}}
%   {\end{threeparttable}\end{center}\end{landscape}}

\newenvironment{lltable}
  {\begin{landscape}\begin{center}\begin{ThreePartTable}}
  {\end{ThreePartTable}\end{center}\end{landscape}}

  \usepackage{ifthen} % Only add declarations when endfloat package is loaded
  \ifthenelse{\equal{\string man}{\string man}}{%
   \DeclareDelayedFloatFlavor{ThreePartTable}{table} % Make endfloat play with longtable
   % \DeclareDelayedFloatFlavor{ltable}{table} % Make endfloat play with lscape
   \DeclareDelayedFloatFlavor{lltable}{table} % Make endfloat play with lscape & longtable
  }{}%



% The following enables adjusting longtable caption width to table width
% Solution found at http://golatex.de/longtable-mit-caption-so-breit-wie-die-tabelle-t15767.html
\makeatletter
\newcommand\LastLTentrywidth{1em}
\newlength\longtablewidth
\setlength{\longtablewidth}{1in}
\newcommand\getlongtablewidth{%
 \begingroup
  \ifcsname LT@\roman{LT@tables}\endcsname
  \global\longtablewidth=0pt
  \renewcommand\LT@entry[2]{\global\advance\longtablewidth by ##2\relax\gdef\LastLTentrywidth{##2}}%
  \@nameuse{LT@\roman{LT@tables}}%
  \fi
\endgroup}


\ifxetex
  \usepackage[setpagesize=false, % page size defined by xetex
              unicode=false, % unicode breaks when used with xetex
              xetex]{hyperref}
\else
  \usepackage[unicode=true]{hyperref}
\fi
\hypersetup{breaklinks=true,
            pdfauthor={},
            pdftitle={Is vowel duration an acoustic cue for L2 speakers producing English plosives?},
            colorlinks=true,
            citecolor=blue,
            urlcolor=blue,
            linkcolor=black,
            pdfborder={0 0 0}}
\urlstyle{same}  % don't use monospace font for urls

\setlength{\parindent}{0pt}
%\setlength{\parskip}{0pt plus 0pt minus 0pt}

\setlength{\emergencystretch}{3em}  % prevent overfull lines


% Manuscript styling
\captionsetup{font=singlespacing,justification=justified}
\usepackage{csquotes}
\usepackage{upgreek}

 % Line numbering
  \usepackage{lineno}
  \linenumbers


\usepackage{tikz} % Variable definition to generate author note

% fix for \tightlist problem in pandoc 1.14
\providecommand{\tightlist}{%
  \setlength{\itemsep}{0pt}\setlength{\parskip}{0pt}}

% Essential manuscript parts
  \title{Is vowel duration an acoustic cue for L2 speakers producing English
plosives?}

  \shorttitle{L2 production of English plosives}


  \author{Ana N. Rinzler\textsuperscript{1}}

  % \def\affdep{{""}}%
  % \def\affcity{{""}}%

  \affiliation{
    \vspace{0.5cm}
          \textsuperscript{1} Rutgers University, the State University of New Jersey\\
          \textsuperscript{}   }

  \authornote{
    Cognitive Psychology
    
    Rutgers Center for Cognitive Science
    
    I would like to thank Dr.~Terry Kit-fong Au from the University of Hong
    Kong for permitting us to conduct acoustic analyses on production data
    from her research.
    
    Correspondence concerning this article should be addressed to Ana N.
    Rinzler, 1498 State Rte. 28 West Hurley, NY 12491. E-mail:
    \href{mailto:anb136@psych.rutgers.edu}{\nolinkurl{anb136@psych.rutgers.edu}}
  }


  \abstract{Enter abstract here. Each new line herein must be indented, like this
line.}
  \keywords{keywords \\

    \indent Word count: X
  }




  \usepackage{tipa}

\usepackage{amsthm}
\newtheorem{theorem}{Theorem}[section]
\newtheorem{lemma}{Lemma}[section]
\theoremstyle{definition}
\newtheorem{definition}{Definition}[section]
\newtheorem{corollary}{Corollary}[section]
\newtheorem{proposition}{Proposition}[section]
\theoremstyle{definition}
\newtheorem{example}{Example}[section]
\theoremstyle{definition}
\newtheorem{exercise}{Exercise}[section]
\theoremstyle{remark}
\newtheorem*{remark}{Remark}
\newtheorem*{solution}{Solution}
\begin{document}

\maketitle

\setcounter{secnumdepth}{0}



\section{Methods}\label{methods}

In this study acoustic analyses were conducted on Cantonese speakers'
productions of English phonological minimal word pairs with voiced (i.e.
/b d g/) and voiceless (i.e. /p t k/) plosives in coda position. The
production of the word \enquote{got} was excluded from this analysis as
it was the only word that did not have a minimal pair. For all other
productions, PRAAT was used to measure the duration of the vowel.
Measuring vowel duration was motivated by evidence that vowel length is
an acoustic cue that English speakers use to distinguish whether the
following plosive is voiced or voiceless. For instance, the duration of
the vowel preceding a voiced stop is typically longer than the duration
of a vowel preceding a voiceless stop (Charles-Luce, 1985; House and
Fairbanks, 1953; Peterson and Lehiste, 1960; House, 1961; Umeda, 1975;
Klatt, 1976). Please note that the production data analyzed in this
study was collected and generously provided by Dr.~Terry Kit-fong Au,
from the University of Hong Kong.

\subsection{Participants}\label{participants}

There were a total of 36 University students from the University of Hong
Kong. 18 of the participants were in the training group (33\% men), and
18 of the participants were in a wait-list control group (28\% men).

\subsection{Material}\label{material}

The following analyses are based on productions of phonological minimal
word pairs with voiced and voiceless plosives in coda position. The
vowel duration from the following voiced words were analyzed:/b\ae d,
bæg, k\ae b, k\textturnv b , d\textopeno g , f\ae d, fid, p\textsci g,
t\ae b/. The following voiceless words were analyzed: / b\ae t, b\ae k,
k\ae p, k\textturnv p, d\textscripta k, f\ae t, fit, p\textsci k,
t\ae b/. Only \enquote{post-training} productions were analyzed. For the
wait-list control particpants, this was the second time that they
produced above words (e.g.~they did receive training in between the
first and second times that they produced these words). However for
trained participants, these productions represent the second time that
they produced these words after they received training.

\subsection{Procedure}\label{procedure}

Participants in Terry Au's (ms) study participated in a 4 - 6 week
training program compromised of comprehending and producing English
phonological minmial word pairs. Not all of the words that were used in
training were used in production. See \emph{Appendix A} for full list of
words, as well as which words were used in training, and which were not.
The productions were then sent to our lab for acoustic analyses.

The software PRAAT was used to conduct acoustic analyses. Textgrids were
created from the .wav sound files in order to mark the beginning and end
of the vowel boundary. Utilizing Sennheiser HD 555 headphones, the
beginning of the vowel was marked with the \emph{wav} method (cite) and
the end of the vowel was marked with the \emph{F2} method (cite). All
boundaries were marked at the zero-crossing line. Measurements at
present, were only taken by one researcher. Thus, future
cross-validation through concordance rates is required. PRAAT scripting
was then used to export vowel duration measurements.

\subsection{Data analysis}\label{data-analysis}

We used R (Version 3.4.3; R Core Team, 2017) and the R-package
\emph{papaja} (Version 0.1.0.9709; Aust \& Barth, 2018) for all our
analyses. Data from the production task were analyzed using a general
linear mixed-effects model using the lme4 package (1.1-10 in R 3.2.2).
The criterion variable was \emph{vowel duration} which was convereted to
milliseconds and normalized for speaker. There were two predictors which
were fixed factors: (1) training \emph{trained/untrained} and voicing
(2) \emph{voiced/unvoiced}. Both factors were cateogrical and were sum
coded. For the training variable, \emph{trained} was assigned a 1, and
\emph{untrained} was assigned a 0; while \emph{voiced} was assigned a 1
and \emph{voiceless} was assigned a 0. Two new columns in the data frame
were generated to represent the sum variables of the training and the
voicing variables. The variable participant was treated as a random
effect as each participant had multiple productions (i.e.~each
participant produced each of the 36 voiced and voiceless words). Visual
inspection of the Q-Q plots and plots of residuals against fitted values
revealed that the assumptions of normality and homoscedasticity were in
tact.

\section{Results}\label{results}

\section{Discussion}\label{discussion}

\newpage

\section{References}\label{references}

\begingroup
\setlength{\parindent}{-0.5in} \setlength{\leftskip}{0.5in}

\hypertarget{refs}{}
\hypertarget{ref-R-papaja}{}
Aust, F., \& Barth, M. (2018). \emph{papaja: Create APA manuscripts with
R Markdown}. Retrieved from \url{https://github.com/crsh/papaja}

\hypertarget{ref-R-base}{}
R Core Team. (2017). \emph{R: A language and environment for statistical
computing}. Vienna, Austria: R Foundation for Statistical Computing.
Retrieved from \url{https://www.R-project.org/}

\endgroup






\end{document}
